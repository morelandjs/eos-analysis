\documentclass{article}
\usepackage{amssymb}
\usepackage{hyperref}
\newcommand\Bn[1]{{\rm BN}_{#1}}
\newcommand\In[1]{{\rm IN}_{#1}}
\begin{document}

How to generate spline basis

Let \(\{\Bn1,\Bn1,\Bn1,\In1,\In2,\Bn2,\Bn2,\Bn2,\Bn2\}\) be a sequence
constructed as follows: repeat the lower boundary knot 3 times,
followed by the two internal knots in ascending order, and the upper
boundary knot repeated four times. Note that in defining the sequence,
we need to sort the knots first as an ascending order in Temperature
labelled $x$.  Let us denote these values of temperature of the 
sequence as \(t_i, i=1,\ldots,9\).

To construct the spline basis is a recursive process

Level 0:

Define the eight B-spline basis functions at level 0 as \(B_{i,0} (x) = 1\) 
for \(i=1,\ldots,8\) and iff \(t_i \leq x < t_{i+1}\).  \(B_{i,0} (x) = 0\)  otherwise.
Note that this condition means that \(B_{1,0}\), \(B_{2,0}\),
\(B_{6,0}\), \(B_{7,0}\), \(B_{8,0}\) are identically 0. If some of
the knots become coincident, more of these will be identically 0.  The rest
of them are piecewise constants 1 in the \(i^{\rm th}\) inter-knot
interval, 0 elsewhere.

The recurrsion is as follows: Given the spline basis at level
\(k-1\), we define the spline basis at level \(k\) as follows:
\[
B_{i,k}(x) = \frac{x-t_i}{t_{i+k}-t_i} B_{i,k-1}(x) + 
               \frac{t_{i+k+1}-x}{t_{i+k+1}-t_{i+1}} B_{i+1,k-1}(x)\,.
\]
Note that if and only if both \(B_{i,k-1}\) and \(B_{i+1,k-1}\) are
identically zero, then \(B_{i,k}\) is identically 0.  Moreover, the
basis splines of level \(k\) are piecewise degree \(k\)
polynomials. Finally, \(i\) ranges from 1 to \(8-k\).  A little work
shows that unless knots coincide, only the \(k^{\rm th}\) derivative
is discontinuous at the internal knots.

Unless some knots coincide, the basis splines that are identically zero at
higher levels ($k \leq 3$) are \(B_{1,1}\), \(B_{6,1}\), \(B_{7,1}\), and
\(B_{6,2}\).

The level 3 basis splines (there are five of them) are the B-splines
we are using.  All of them are 0 at \(t_1\), and only \(B_{1,3}\) has
a non-zero derivative there.  

The table we are sending contains 17 numbers per bootstrap sample

(0) Bootstrap Sample Number within quotes
C1) Boundary knot position low  (130)
C2) Boundary knot position high  (610)
C3) Position of first knot
C4) Position of second knot

C5) hrg slope at T=130
C6) hrg value at T=130
C7-C10) Coefficient of spline 2-5
C11) Coefficient of $1/N_\tau^2$
C12-C15) Coefficient of $1/N_\tau^2$ * Spines 2-5
C16) Coeff of $(T-130)/N_\tau^2$

The formula for I(T) is (C1-C4 determine the splines)

\begin{eqnarray}
\cr I(T) &=& C6 + C5 * (T-130) \\ \nonumber
       &+& C7 * B_{2,3}(T) \\ \nonumber
       &+& C8 * B_{3,3}(T) \\ \nonumber
       &+& C9 * B_{4,3}(T) \\ \nonumber 
       &+& C10 * B_{5,3}(T) \\ \nonumber
       &+& C11 / N_\tau^2 \\ \nonumber
       &+& C12* B_{2,3}(T) / N_\tau^2 \\ \nonumber
       &+& C13* B_{3,3}(T) / N_\tau^2 \\ \nonumber
       &+& C14* B_{4,3}(T) / N_\tau^2 \\ \nonumber
       &+& C15* B_{5,3}(T) / N_\tau^2 \\ \nonumber
       &+& C16* (T-130)    / N_\tau^2 \nonumber
\end{eqnarray}

If you are using R -- you dont have to think and calculate the $B_{i,3}$ 
using the recursion formula. 

We are sending you the R program that generates, for each bootstrap
sample defined by the position of the 4 knots (two boundary and two
internal), the values of the 5 splines (our final fit case) at
prescribed values of $T$ (edit values in the do loop
seq(from=7,to=15,by=2) to define the $T$ range and step size).

Once you have that and the table of coefficients, you can 
construct the value of the trace anomaly at any given $T$ by 
using the above formula with $B_{i,3}$ replaced by its value 
generated by R


R code for getting values of the 5 splines at a given $T$:

The following R code shows how to
tabulate the values of these five splines between \(x=7\) to 15 in steps of
2 if the boundary knots are at 5 and 20 and the internal knots are at
8 and 10.
\begin{verbatim}
> library('splines')
> x = seq(from=7,to=15,by=2)
> cbind(x,bs(x,knots=c(8,10),Boundary.knots=c(5,20)))
      x         1          2          3           4     5
[1,]  7 0.4474074 0.48000000 0.03555556 0.000000000 0.000
[2,]  9 0.0200000 0.70944444 0.26708333 0.003472222 0.000
[3,] 11 0.0000000 0.40500000 0.50625000 0.087750000 0.001
[4,] 13 0.0000000 0.19055556 0.51041667 0.272027778 0.027
[5,] 15 0.0000000 0.06944444 0.36458333 0.440972222 0.125
> 
\end{verbatim}

\end{document}

